\documentclass[11pt]{article}
\usepackage{fullpage}
\usepackage{times}
\usepackage{url}

\newcommand {\doublespace} {\addtolength{\baselineskip}{.5\baselineskip}}
\newcommand {\singlespace} {\addtolength{\baselineskip}{-.333\baselineskip}}

\begin{document}
%\pagestyle{empty}
\renewcommand{\thepage}{DMP-\arabic{page}}
\setcounter{page}{1}

\def\myskip{3ex}

\section*{Data Management Plan}

%This plan articulates how sharing of the primary data generated during
%the course of the project for this proposal will be implemented.

\subsection*{Expected Data}

The expected data to be generated during the course of this project are:

\begin{enumerate}
  \item Technical papers describing our research.
  \item Code for the proposed phishing page detection browser plug-in.
  \item A dataset of logos and image maniplations.
  \item Data from user studies.
  \item Data from the RIT phishing training.
  \item Security incident reports and logs of compromised user account statistics from the RIT IT department.
  \item Course materials, such as lectures and slides, homework assignments, and projects.
\end{enumerate}

\subsection*{Data Retention and Storage}
All data generated by this project will be retained for a minimum of
three years after the conclusion of the award, including data that are
not specifically disseminated as described in the following section. We
will use file storage services provided by RIT and department IT
departments to ensure secure preservation of data during the retention
period.

Data collected in user studies, RIT phishing training, and security
incident reports and logs will be encrypted to prevent exposures. In our
user studies, no identifiers will be kept and only basic demographic
information will be associated with the data.

\subsection*{Data Dissemination}
Technical papers generated during the course of the project will be
published in academic journals and conference proceedings. Papers
published in venues without archival proceedings, as well as technical
reports not otherwise published, will be disseminated via arXiv.

The phishing page detection browser plug-in will be released under an
Open Source License (as defined by the Open Source Initiative) and will
be published in an open source project repository, such as SourceForge,
GitHub, or Google code project hosting.

Data produced from the proposed research activities will be made
available either publicly on PI Wright's website or, for sensitive data,
available upon request by other researchers. Exceptions to this are the
RIT phishing training data, which cannot be shared due to RIT policies
on students and employees, and security incident reports and logs, which
are sensitive for RIT's security posture. For these data sets, we will
work with RIT's Global Risk Management team to find ways for other
researchers to access the data, perhaps in conjunction with the PIs.

Shared data will be stored in standard formats, such as text files and
XML, whenever it is possible to do so. If binary formats are necessary
(e.g., for saving space), utilities for converting this data into
human-readable formats will be made available.

Curricular materials such as lecture notes and slides will be made
available via PI Wright's website.
\end{document}
