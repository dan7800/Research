\if false
\section{Roles of the PIs}
PIs Kazman, Cai, Mirakhori, and Ryoo already have a long and successful history of collaborating, both in general and on NSF-supported research (Awards 1140300 and 1065242).  Our collaboration over the past five years has resulted in two journal papers, three conference papers, one book chapter, a tool demo paper, one workshop paper, and one pending patent. In addition, three other papers are currently under submission.  

For the research being proposed here we expect that our roles and relationships will remain largely as follows.  PIs Cai and Mirakhori with the help of their graduate students will lead the design and development of the detection techniques and their underlying theoretical foundations, as well as empirical analysis of the detection technique's primary outputs. PIs Kazman and Ryoo, with the help of their graduate students, will lead the development of the tactics realization ontology and application AAFS in real-world architectural analyses, focusing on empirical studies of the manual and automated aspects of refactoring architectural hotspots for security.  

However this division is simply a matter of leadership: all PIs expect to continue their history of close collaboration on {\em all} research questions, and all expect to co-supervise or be on the PhD committee of one another's students.   


\section{Management Plan}
Our collaboration has already been successful due to a number of factors: we have a practice of engaging in frequent teleconferences and web-conferences (the entire research group meets once a week, and PIs Kazman, Cai, Mirakhori, and Ryoo speak and email regularly in between those weekly meetings), frequent group status update emails, and regular face-to-face meetings (at conferences and workshops and at our respective locations).  In addition we maintain shared repositories and shared tools.  Kazman is also currently co-supervising three of Cai's PhD students.   We have no reason to expect that any of this will change in our collaborations for the proposed research.

The project management will build on our current successes and models.  We intend to continue our tradition of weekly group web-conferences involving all PIs and all students, and quarterly face-to-face meetings, typically lasting 2-3 days. Not all of these meetings are at our home universities; we expect to continue our successful practice of meeting, and working, at national and international conferences such as ICSE, WICSA, FSE, and other appropriate venues as these save on time (since we are already attending those conferences) and save on travel funds.  Furthermore, we will continue our use of cloud-based repositories that allow us to easily share data, papers, and other research results.

To support these collaborations and mechanisms we have budgeted regular conference trips, both national and international, and travel to one another's locations.  We have also budgeted a small amount of funds for computer purchases and software services (to pay for the shared cloud-based data repositories and for the occasional purchase of software tools to support our research).  

Finally, our research is, in the end, empirically grounded.  We can not confidently validate our results unless we work closely with industrial partners.  And so we expect that we will make occasional trips to work together, with industrial partners, at their locations.  We already have built substantial relationships with industry, and we expect to deepen and broaden this  external collaboration in our proposed research program over the next four years.
\fi

\section{Timeline Highlights}
The projected timeline of our project is shown in Figure~\ref{fig:timeline}. Most of the tasks overlap since the detection of tactics and vulnerabilities, anti-pattern detection and solution, as well as their evaluation have to explored side-by-side.  Our focus in the first two to three years is to implement and extend the key abstractions and algorithms.  In the latter portion of the grant, we will focus more on empirical validation, the creation of data acquisition infrastructure, the creation of sharable empirical results and data sets,  the construction and usability evaluation of AAFS, and technology transfer.  


\begin{figure*}[ht]%{l}{.5\textwidth}
\centering
	\includegraphics[width=1\textwidth]{figures/timeline.png}
\caption{Project Timeline }
\label{fig:timeline}
\end{figure*}


