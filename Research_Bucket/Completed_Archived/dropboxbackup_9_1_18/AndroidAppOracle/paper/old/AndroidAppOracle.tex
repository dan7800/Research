%  \documentclass{sig-alternate}
%\documentclass[conference]{IEEEtran}
\documentclass{sig-alternate-05-2015}


\usepackage{cite}
\usepackage{url}
\usepackage{color}
\usepackage{balance}
\usepackage{caption}


\newcommand{\todo}[1]{\textcolor{cyan}{\textbf{[#1]}}}
\newcommand{\sam}[1]{\textcolor{green}{{\it [Sam: #1]}}}
\newcommand{\dan}[1]{\textcolor{blue}{{\it [Dan: #1]}}}


\newif\ifisnopii
%\isnopiifalse % Hide Info
\isnopiitrue  % Show Info

\begin{document}

%	An Oracle of Vulnerable Android Apps for Education
%	An Educational Oracle of Vulnerable Android Apps

\title{XXXXX}

\numberofauthors{1}
\ifisnopii % turn on/off pii
\author{
\alignauthor
Daniel E. Krutz, Hussein Talib and Samuel A. Malachowsky 	\\
	\affaddr{Software Engineering Department}\\
       	\affaddr{Rochester Institute of Technology}\\
	\affaddr{1 Lomb Memorial Drive}\\
	\affaddr{Rochester, NY, USA}\\
       \email{\{dxkvse,hat6622,samvse\}@rit.edu}
       \alignauthor
} % Must not be a space above this
\else % turn on/off pii
\author{
\alignauthor
XXX X XXX\\
       \affaddr{xxxxx}\\
       \affaddr{xxxx, xx, xxx}\\
       \email{xxxxx@xxx.xxx}
}
\fi % end turn on/off pii





\maketitle
\begin{abstract}

The mobile revolution has allowed anyone with a basic understanding of application (app) development to upload their apps to an app store, and make them available to millions of potential users. With this great power comes great danger as inexperienced developers have the capability to create vulnerable apps which can affect millions of users. Moreso, even experienced developers make mistakes due to the challenging nature of creating secure software.

Recent publicized vulnerabilities from even the most popular app store, Google Play, have ranged from.......

Developers created vulnerable apps for a wide range of reasons, some of which include ignorance of how to create secure apps, simple developer mistakes, or a lack of understanding of the importance of secure app development.



In order to help educate developers about not only how to create secure apps, but the importance of secure app development, we have created a public educational set of Android apps. Each example contains a clear demonstration of the negative ramifications of the vulnerability, steps to repair the vulnerability, and then steps to ensure that it has been resolved.

We have two primary goals for the project. The first is that we hope to educate Android developers about the specific example vulnerabilities in our study. Secondly, we would like to demonstrate the importance of security on a more general level to a diverse set of Android developers from an experience perspective. Our goal is to get developers, from all experienced levels, more interested in security and to see how much of an important topic it is.

Our oracle and all project resources are publicly available on our website: \ifisnopii {\url{XXCreate URLXXX}} \else \url{http://hiddenToKeepAnonymous. } \fi







%	Maybe sell as a project which will keep growing. Others can contribute to.

\end{abstract}


%%% FIX ALL OF THESE
\category{K.X.X}{Computers and Education}{XXXXX}[XXXXXX]

\terms{XXXX, XXXX}

\keywords{XXXX, XXXX}



\section{Introduction}

Android is the world's most popular mobile OS~\cite{OSMarketShare_URL} with over 1.8 million apps available from Google Play alone~\cite{statistica_url}. Unfortunately, this popularity comes at a steep cost, since Android is the most frequent target of security related attacks, and whose apps routinely suffer from serious security vulnerabilities.These vulnerabilities can range from 

small leaks insignificant data, to the user to.... 

\cite{Android_cv_url}

\todo{cite} Understanding proper protection methods, and defensive coding practices is paramount for any developer in creating apps which are secure and better protect the user's privacy \todo{cite}. 

https://www.cvedetails.com/vulnerability-list/vendor_id-1224/product_id-19997/Google-Android.html

% http://androidvulnerabilities.org

A primary driving factor of the mobile revolution is the fact that anyone with basic programming experience can create an app, upload it to their platform's app store, and have it used by millions. However, this is a double edge sword since  inexperienced developers can make mistakes which can seriously impact the security of an app. In fact, even the most experienced mobile developers can make mistakes which hinder an app's security as well \todo{cite}.

%% What is the %%% of apps which contain vulnerabities. What are some of these vulnerabilities. What is some information which states that education is a good way to limit vulnerabilities?

%%% Show stats about how users are not prepared to securily develop apps.
%	Creating relevant vulnerabilities is not easy. Can take time.

%% explain difference between abuse and misuse cases. Lead this into other areas of the paper


In order to help address this shortfall for both abuse and misuse cases, we can created a publicly available, educational oracle of Android vulnerabilities. Each example contains the following:


%	What are some problems with Android apps. Who is creating them etc...

%	What we did
%	Introduce how it can be used
%	What are the deficincies solved in this paper



\begin{enumerate} 
	\setlength{\itemsep}{.8pt} %Cut down on spacing for the different items in the list
    	\setlength{\parskip}{0pt} %Cut down on spacing for the different items in the list
    	\setlength{\parsep}{0pt}  %Cut down on spacing for the different items in the list
   
	\item A clear example of how to re-create the vulnerability.
	\item A definition of why the vulnerability is detrimental to an app along with background about the issue.
	\item Clear steps of how to repair the vulnerability.
	\item Steps of how to demonstrate that the vulnerability has been successfully repaired.

\end{enumerate}

Our goal for the project is for other instructors to use these activities in both their mobile and security related courses. Instructors may choose to use all components, or even only a single component in their course. Each example or module is independent of one another, and vary from basic vulnerabilities, to ones which are oriented to more advanced developers. Since each activity contains robust instructions for its use, individuals who are hoping to learn more about mobile development and security may use these activities as well.

%% Where has this example been used so far?

%% 

% The rest of the paper is organized as follows: In Section~\ref{sec: relatedworks} 


\section{Oracle Creation}
\label{sec:oraclecreation}

%	Impossible to create items for all situations and all possible vulnerabilities 
%	How we selected the vulnerabilities. What criteria did we use?
%		Which were the most popular, new and challenging. Also ones which we thought would be interesting to students
%	Goal was to have easy to medium vulnerablities
%	Primarily developed


\section{Created Vulnerable Apps}
\label{sec:vulnerableApps}


%	List the vulnerabilities that we created. 
%	Write a bit about each of the vulnerabilities
%	What type of vulnerability does each belong to? Do they represent different groups?
%	Maybe create a table with the name of the vulnerability, along with its target difficulty level
%	What were some of the challenges in our app creation process?


We currently have \todo{XXX} apps in our oracle, although we expect this list to grow on a regular basis.


\todo{update \& expand this list}



\begin{enumerate} 
%	\setlength{\itemsep}{.8pt} %Cut down on spacing for the different items in the list
 %   	\setlength{\parskip}{0pt} %Cut down on spacing for the different items in the list
%    	\setlength{\parsep}{0pt}  %Cut down on spacing for the different items in the list
   
   
   
%	\item \textbf{XXX}: XXXX



	\item \textbf{AdLibraries: }XXXX


	\item \textbf{Android Javascript: }XXXX




	\item \textbf{Broadcast: } XXXX



	\item \textbf{Activities Access: } XXXX

	\item \textbf{Content Providers: } XXXX


	\item \textbf{Data Storage: } XXXX

	\item \textbf{DataOverHTTP: } XXXX


	\item \textbf{DOS: Denial of Service:} XXXX

	\item \textbf{Intent} XXXX

	\item \textbf{XML:} XXXX 




\end{enumerate}








\subsection{Sample App: XXXX}

In order to best demonstrate our set of vulnerable apps and how it may be best used, we will provide information on one of the apps created our oracle.



\section{Usage Instructions} % Find a better title for this
\label{sec:usage}





%% Sell that the project is going to keep growing. That it is not complete
\section{Future Growth} % Find a better title for this
\label{sec:futuregrowth}


% - On webpage create a mechanism for people sharing their information with us. Show that we are eliciting information
%	Instructions how to provide information... pull request????
%	Create a template which they can use to provide data
%	


\section{Related Work}
\label{sec:relatedworks}

%	Other work that supports security/mobile education
%	Work that has identified vulnerabilities
%	How have vulnerabilities in Android/mobile development been addressed?
%	



\section{Limitations \& Future Work}
\label{sec:futurework}


%	Create workshops for children, especially underrepresented groups. Goal will not only be to educate these students on proper security practices, but also get them important in computing, 
%	Keep adding apps an examples to the oracle
%	Create screencasts and instructional videos for each of the vulnerabilities
%	Apply to iOS. Will share.some of the same type of vulnerabilities, but will be some new types as well
%		What are some iOS vulnerabilities
%	Create network with other institutions allowing them to contribute to our project
%	



\section{Conclusion}
\label{sec: conclusion}

We have created a publicly available instruction set of vulnerable Android apps which includes 10 groups of vulnerabilities. Our goal is for instructors to use these activities in a diverse set of courses, and may be used by a diverse set of skill levels ranging from beginning level developers, to more advanced levels. All course material may be found on our project website: \textbf{\url{XXXXX}}


\section*{Acknowledgements}
\ifisnopii % turn on/off pii
This work is partially sponsored by a SIGCSE Special Projects Grant. % Maybe we should leave this in regardless. It makes it more likely that the paper will be accepted

\else % turn on/off pii
Author and funding acknowledgments hidden for review anonymity.
\fi % end turn on/off pii

\balance 
\bibliographystyle{abbrv}
\bibliography{AndroidAppOracle}



% That's all folks!
\end{document}



%%%% Todo
%	Probably create some screenshots of the apps being used
%	Create a new oracle & website (& Github) we can use for the project: https://github.com/dan7800/VulnerableAndroidAppOracle
% If switching to TOE, change format and structure of the paper
% Create oracle website


%% Possible venues:
%	SIGSCE - 6 pages: http://sigcse2016.sigcse.org/authors/papers.html	 - Due date: 9/1?
%	Other venues: 
%	 ITiCSE: https://eventi.unibo.it/iticse2017
%	 ICER: Due April
% 	ACM Inroads or 
% 	(1) ACM Transactions on Computing Education: Submit 1st day of the month - If it does not get in here, then submit to SIGCSE (33, 41,16 pages....)
%		A recent paper: 



%%% Project URLs
%	

