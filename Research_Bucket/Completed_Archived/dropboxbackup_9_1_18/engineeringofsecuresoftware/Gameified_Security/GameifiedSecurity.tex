\documentclass[conference]{IEEEtran}

\usepackage{cite}
\usepackage{listings}
\usepackage{booktabs}
\usepackage{color}
\usepackage{array}
\usepackage{subfigure}
\usepackage{balance} % Add this back in. Probably needed during camera ready.
\usepackage{url}
\usepackage{times} % Used for formatting formatting url footnotes
\urlstyle{same} % Used for formatting formatting url footnotes
\usepackage{caption} % Used for formatting formatting url footnotes
\usepackage{listings}


\newcommand{\todo}[1]{\textcolor{cyan}{\textbf{[#1]}}}
\newcommand{\andy}[1]{\textcolor{red}{{\it [Andy says: #1]}}}
\newcommand{\dan}[1]{\textcolor{blue}{{\it [Dan says: #1]}}}
\newcommand{\sam}[1]{\textcolor{green}{{\it [Sam says: #1]}}}

\begin{document}

\title{XXXXXXXXXXXX}

\author{\IEEEauthorblockN{Andrew Meneely, Daniel E. Krutz, and Samuel A. Malachowsky}
Rochester Institute of Technology\\
\{axmvse, dxkvse,samvse\}@rit.edu

}


\maketitle
\begin{abstract}

Abstract

\end{abstract}

\begin{IEEEkeywords} Software Security, Software Engineering, Computing Education\end{IEEEkeywords}


%\todo{define the categories and fix the keywords \& terms}
% A category with the (minimum) three required fields
%\category{H.4}{Information Systems Applications}{Miscellaneous}
%A category including the fourth, optional field follows...
%\category{D.2.8}{Software Engineering}{Metrics}[complexity measures, performance measures]


%\category{K.3.2}{Computers and Education}Computers and Information
%Science Education- Computer science education; Curriculum

%\terms{xxx, xxxx, xxxx}

%% My best show at keywords
%\keywords{xxxxx, xxxxxx, xxxx}

\section{Introduction}

Introduction



% At the Rochester Institute of Technology, we created an upper division Engineering of Secure Software applications course to help students understand how to incorporate proper security protection practices when designing, creating, and maintaining software. Some course topics include defensive coding practices, deployment \& distribution strategies, vulnerability assessments, and threat modeling. In this course, we created an activity to help acclimate students with how to understand, protect, and recognize insider threats. Students teams are formed and students are given the task of designing a small application and planning for proper security. One student from each team is quietly pulled aside and told that they are the insider threat or~\emph{mole} for their team. Their goal is to have their team design an application containing a vulnerability which the mole would be able to later use. After the activity, the moles are revealed and a discussion takes place regarding how the moles were able to create the vulnerability, if the team recognized this vulnerability, and what could have been done to prevent this insider threat.
% Probably a good idea to add more to the discussion topics?


% Insiders will learn how malicious insiders work and this "thinking like an insider" will help to prepare them for this threat in future when faced in the real world


The rest of the paper is organized as follows: Section~\ref{sec: aboutcourse} describes the course including learning objectives. Section~\ref{sec: activity} discusses how the activity was conducted. Section~\ref{sec: studentfeedback} provides student feedback about the project including quotes and post activity survey analytics. Section~\ref{sec: relatedwork} presents some related works and Section~\ref{sec: futurework} discusses possible future work and improvements to the activity. Section~\ref{sec: conclusion} provides concluding remarks about our work. \todo{update entire section}



\section{About the course}
\label{sec: aboutcourse}

\todo{reword entire section. This is right out of the FIE paper}
Primarily comprised of upper division Software Engineering students, the Engineering of Secure Software
course\footnote{\url{http://www.se.rit.edu/~swen-331/}} was created in 2012 and is focused on instructing students in the proper practices of design and creating secure software. The only prerequisite is the Introduction to Software Engineering course in which students are introduced to core concepts in software engineering such as development methodologies, team work in software development, basic testing principles, and software design.

The course has a primary learning outcome of preparing students to mitigate security threats in software systems and processes. The focus is on proper methods of designing, developing, testing, and maintaining secure software. While the course is language-agnostic and focuses on principles and practices, specific tools and technologies are used to reinforce the learning objectives of the course. For instance, Microsoft's SDL Threat Modeling Tool\footnote{\url{www.microsoft.com/security/sdl/adopt/threatmodeling.aspx}} is used to instruct students on the proper methods of designing the architecture of a secure system. Specific Java-based examples are used to demonstrate SQL injection attacks, log overflow attacks, hashing and salt, and path traversal exploits. Short Vulnerability-of-the-Day activities serve to introduce students to real world examples of exploits and demonstrate the importance of software security~\ \cite{MeneelyICSESEE2013}. Students work in small teams on several course projects including the creation of a web fuzz testing tool and a case study which examines a real-world software project for vulnerabilities.

While we do not expect all students taking the course to become security experts, our goal is to instill fundamental principles of secure software development in the students while demonstrating its importance in the real world. Students are graded on several criteria such as three exams, several short projects, and brief in-class activities. Class size is typically 25-35 students and is a required course in the Software Engineering major.

In the course, we also discuss several ways of protecting against insider threats. While there is no easy or simple silver bullet protection mechanism against insider threats, there are some best practices which may be used to help alleviate this risk. Some of these protection practices include properly screening potential employees, implementing end point data leak protection, monitoring databases \& sensitive records, and the proper use of rights management systems~\cite{Insiderthreat_protection_url}.


% ? Should more be added to this?
%       Could break this down on a week by week basis

\section{Game Activity} % Find a better title for this
\label{sec: activity}

Activity Intro

\subsection{activity}



\subsection{How students did}



\subsection{Post Activity Discussion and Goals}
%\label{sec: discussion}

Discussion

\section{Student Feedback}
\label{sec: studentfeedback}

\todo{update this entire section}
Students have expressed a significant amount of satisfaction in this activity and it has contributed to their overall satisfaction with the course. At the conclusion of the project, students are asked to submit an anonymous survey asking them to provide feedback regarding the project. Some of the questions were based upon the Likert scale, while other asked students to provide written feedback. Several of these questions and student responses are shown in Table~\ref{table:studentfeedback}. The survey has been posed to students in the last three course offerings, all of which have used this activity component. A total of 68 students from these sections have chosen to respond.

\begin{table*}[t]
\caption{Student Responses}
\centering
    \begin{tabular}{ l | c | c | c | c | c     }

	\bfseries  & \bfseries Strongly Agree & \bfseries Agree & \bfseries Undecided & \bfseries Disagree  & \bfseries Strongly Disagree \\ \hline \hline

	 You enjoyed the activity & x & x & x & x  & x \\ \hline
	
    \end{tabular}

\label{table:studentfeedback}
%\vspace{-0.3in}
\end{table*}
\todo{make sure to update everything about the table}



%These results indicate that the vast majority of students not only enjoyed the activity, but would also recommend it to a classmate as well. Additionally, most students felt that it reassembled a project which they were likely to encounter in the real world and were similar to tasks they were asked to complete while on cooperative internships.


The following are samples of written feedback that have been received:

\begin{quotation}
``Blah''
\end{quotation}

This feedback indicates that students not only enjoyed the activity, but felt that it was an effective learning mechanism as well.

\section{Related Work}
\label{sec: relatedwork}

Related Work


%% Look at citations from other similar works we've done

\section{Future work}
\label{sec: futurework}

Future Work.

\section{Conclusion}
\label{sec: conclusion}

Conclusion




%\end{document}  % This is where a 'short' article might terminate
\balance
\bibliographystyle{abbrv}
\bibliography{GameifiedSecurity}

%\balancecolumns
% That's all folks!
\end{document}




% Todo
%	Format to the style of the conference we'd be submitting to

