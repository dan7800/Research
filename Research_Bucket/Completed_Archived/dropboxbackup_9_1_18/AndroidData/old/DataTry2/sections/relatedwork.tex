\todo{See if any need to be added back in from the comments}

% \todo{This section needs to be changed to not just address the permissions gap}
% \todo{make sure this section is really good, I can see people really scrutinized this.}

Android applications have been extensively researched in numerous areas. The topic of reducing the permission gap in Android applications has received a considerable amount of attention recent years. Much of the existing work in this area has dealt with ways of reducing these unneeded permissions and the security vulnerabilities they may lead to. Jeon~\emph{et al.} introduced a framework for creating finer-grained permissions in Android. They believe that the course-grained permissions currently used by Android limit developers by forcing them to choose all of the permissions located in each ``bucket'' when they really only want to add a few of them. This leads to applications having many more permissions than they actually require. The authors believe that finer-grained permissions would lead to only having the needed permissions used by an application, and thus would lead to few vulnerability possibilities~\cite{jeon2011dr}.

Wei~\emph{et al.} studied the evolution of Android to determine if the platform was allowing the system become more secure. They found that the privacy and security in the overall Android system is not improving over time and that the principle of least privilege is not being adequately addressed~\cite{Wei:2012:PEA:2420950.2420956}.

There have been innumerable studies analyzing mobile applications on a large scale. Sarma~\emph{et al.} evaluated several large datasets, including one with 158,062 Android applications in order to gauge the risk of installing the application, with some of the results broken down by genre. However, this work did not analyze the application using the range of static analysis tools which we used. Viennot~\emph{et al.} developed a tool called PlayDrode which they used to examine the source code of over 1,100,000 free Android applications. While the authors examined a very large number of applications, they largely only used existing information which could be gathered from GooglePlay and, while they carried out static analysis on the applications, they examined features such as library usage and duplicated code --- not areas such as security vulnerability levels and overpriviledges, which was a part of our analysis.


While this work represents the largest known empirical analysis of developers allowing overprivileges to occur in Android applications, it is not the first research into developers not following the principle of least privilege. Felt~\emph{et al.} described some common developer errors found using their tool Stowaway including confusing permission names, the use of depreciated permissions, and errors due to copying and pasting existing code~\cite{Felt:2011:APD:2046707.2046779}. In another work, Felt~\emph{et al.} very briefly described some inclinations they had for why developers gave too many permissions to applications, but this was largely based on assumptions and not necessarily data~\cite{Felt:2011:EAP:2002168.2002175}.


%There have been several tools which have been developed to assist in the decision making, permission process for developers. Felt~\emph{et al.} created a tool known as~\emph{Stowaway} which uses a permissions-to-API calls maps in order to statically analyze request permissions in Android applications~\cite{Felt:2011:APD:2046707.2046779}. This tool notes the extra, unneeded permissions requested by the application, along with permissions that should have been requested, but were not. One criticism of this tool is that it may be difficult to determine if a permission is actually used through the use of static analysis.

Permlyzer is a tool which was built to determine where permissions are utilized in Android applications by using a mixture of static and runtime analysis~\cite{6698893}. This is a recently published tool, so it has not yet been discussed or used in a substantial amount of subsequent research. The authors were, however, able to achieve promising results and this may be a powerful tool for assisting in the permissions granting decision process for developers. \emph{PScout} was another tool developed to extract permission specifications from Android applications using static analysis~\cite{Au:2012:PAA:2382196.2382222}. While the authors of this tool were able to achieve promising results, subsequent work has criticized this tool for not being accurate enough, since Android's permissions could be different at runtime --- something the tool is not capable of discovering~\cite{zhang2013vetting}.

Bartel~\emph{et al.} and Wei~\emph{et al.} also discussed some basic, high level discoveries about why developers make these mistakes~\cite{Bartel:2012:ASP:2351676.2351722,Wei:2012:PEA:2420950.2420956}. While these works were beneficial for numerous reasons, no known works to date have explored the question of why developers do not adhere to the principle of least security as consistently as they should.
