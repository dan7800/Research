
%\label{sec: Introduction}

% Most vulnerabiliteis are not due to the phone/OS, but due to apps
% Smart phones are not really phones, they are now supercomputers in our pockets


Current ``smartphones'' are no longer really phones, they are supercomputers we carry around in our pockets which just happen to be capable of placing calls. Most smartphone vulnerabilities are not due to problems with the phone's operating system, but are caused by intentional or unintentional vulnerabilities in the phone's apps. These are created by careless developers leaving vulnerabilities their code of the apps or using libraries containing vulnerabilities, or by developers intentionally creating malware.\todo{cite?}

Vulnerabilities are unfortunately a pervasive aspect of software. Eliminating vulnerabilities is an important, but extremely difficult task. Software developers not only need to be concerned with vulnerabilities in their code, but with 3rd party components as well~\cite{Sun:2014:NPA:2627393.2627396}.  Conversely, user's should have quality, easily accessible information to enable them to make an informed, objective decision about which apps and app versions off the best level of security. Researchers pave the way for new discoveries in both the process of developing secure software, and in understanding why vulnerabilities take place at a technical level. In order to carry out both qualitative and quantitative studies, researchers need quality and accessible data sets in which to analyze. Students need quality data, examples and exercises to educate them about the importance and process of creating secure apps.

While there are numerous tools and techniques of varying degrees of effectiveness for preventing vulnerabilities or finding them in an app currently under development, there are very few mechanisms for analyzing Android apps in the wild, the lifecycle of an app, or of understanding the Android development process.



% A way to help eliminate vulnerabilities is to prevent them in the first place, which can be accomplished through educating students, developers and researchers.


%% What do recent studies find for the amount of vulnerabilities in apps?


% Drive home the need of the paper here

%catchy

%What specific problems do apps suffer from?

%% The reader should be "sold" on the idea of this paper here.

%% Provide some clear examples of how this has been problematic in the real world.
%%		Give examples of what the NSA, DHS would care about
%%		Section should be much like a problem statement. What are the current problems that need to be solved? Motivate the need for our work




%%%% Problem status
% Malicious Android apps rise by 400% http://www.zdnet.com/article/riskiq-claims-malicious-android-apps-up-by-almost-400-percent-on-google-play/

% 97% of mobile malware is on Android - http://www.forbes.com/sites/gordonkelly/2014/03/24/report-97-of-mobile-malware-is-on-android-this-is-the-easy-way-you-stay-safe/

%


 In order to assist students, researchers, developers and general users in understanding apps, their development process, and security, we have created a free website at~\textbf{\url{http://androsec.rit.edu}} which may be used to analyze Android applications.

%%% Really drive home how each group can use the research
Developers can learn from previously developed apps by seeing their mistakes both in terms of code quality and security, along with things other developers have done well. Researchers can analyze the lifecycle of apps at an individual or aggregate level. End users may use this dataset to compare apps to one another in terms of possible security vulnerabilities and code quality. They may even compare different versions of the same app against one another using these same metrics. Various app genres may also be compared in a variety of manners. 

We began by collecting information about over 1,100 Android apps from F-Droid~\cite{fdroid_url}, which is a catalog of free an open source Android applications that contains the public version control systems of these apps. These apps ranged from small, seldom used apps; all the way to some of the more popular Android apps such as ``2048'', ``VLC''  and ``Adblock Plus''. We then analyzed over 4,416 different versions of these apps using several  existing static analysis tools such as Stowaway~\cite{Felt:2011:APD:2046707.2046779}, Androrisk~\cite{androguard_url}, and Sonar~\cite{sonar_qube_url}\dan{remove sonar?}. We also recorded information from 435,680 version control commits including when the commit was made, who made it, and the commit message. All results are available in a publicly accessible SQLite database on our website.

In the following work, we discuss the how the website's data was created, how to use the website, and usage scenarios for classroom activities and education, researchers, developers and general users.



%In the following work, we discuss the how the website's data was created, how to use the website, and some of its uses for classroom activities and education, researchers, developers and general users.



\todo{Put a quote from Thomas Richards somewhere in here?}
% Make sure the paper is really appealing to people in industry

