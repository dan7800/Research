\newcommand{\sam}[1]{\textcolor{red}{{\it [Sam says: #1]}}}

\documentclass{sig-alternate}


\usepackage{epstopdf}
\usepackage{color}
\usepackage[table]{xcolor}
\usepackage{balance}
%\usepackage{auto-pst-pdf}

\newif\ifisnopii
\isnopiifalse % change to false to remove personally identifiable information (pii)

\begin{document}
%
% --- Author Metadata here ---
\conferenceinfo{SIGCSE'15,}{March 4--7, 2015, Kansas City, MO, USA.}
\CopyrightYear{2014} % Allows default copyright year (20XX) to be over-ridden - IF NEED BE.
%\crdata{0-12345-67-8/90/01}  % Allows default copyright data (0-89791-88-6/97/05) to be over-ridden - IF NEED BE.
% --- End of Author Metadata ---



\title{Implementing Project Managers in the Computing Classroom}

\numberofauthors{1}
\ifisnopii % turn on/off pii
\author{
\alignauthor
Samuel A Malachowsky\\
       \affaddr{Rochester Institute of Technology}\\
       \affaddr{Rochester, NY, USA}\\
       \email{samvse@rit.edu}
}
\else % turn on/off pii
\author{
\alignauthor
XXX X XXX\\
       \affaddr{xxxxx}\\
       \affaddr{xxxx, xx, xxx}\\
       \email{xxxxx@xxx.xxx}
}
\fi % end turn on/off pii

\maketitle
\begin{abstract}
Project management is a discipline that spans many industries and has undeniable benefits in its application. Sometimes, however, it can be difficult to convey its importance in the classroom environment. Unfortunately, many project management classes cover the core concepts, but fail to provide students with the opportunity to experience the leadership elements so core to the discipline.

This article describes an innovative approach to using Project Managers in the classroom that has had
measured effects in several areas, including individual student participation, group project disposition, and
in-class presentations. Results have been encouraging, with student feedback indicating positive effects
on interest in the field and application of project management, improved group dynamics, and more
individual participation in the outcome of group projects. The concepts herein have been successfully
implemented with software engineering students, but they could easily be applied to any classroom that
wishes to expand project management instruction beyond simple explanation of concepts, specifically
through the use of a project manager-led group project.
\end{abstract}

\category{K.3.2}{Computers and Education}{Computer and Information Science Education}[curriculum,computer science education]
\category{D.2.9}{Software Engineering}{Management}[productivity, programming teams]
\category{K.6.1} {Management of Computing and Information Systems}{Project and People Management} [management techniques]

\terms{Management, Human Factors}

\keywords{Classroom, Curriculum, Project Management, Software Engineering, Team Development}

\begin{table*}[t]
\centering
\caption{\textit{Process and Project Management} topics by week}
\label{table:topics}
\begin{tabular}{|p{0.25cm}|p{3.25cm}|p{13.0cm}|} \hline
\cellcolor[gray]{0.9}&\textbf{Topic}\cellcolor[gray]{0.9}&\textbf{Details}\cellcolor[gray]{0.9}\\ \hline
1&Course Introduction&Course overview, what a project is, why process is important, basic project terms\\ \hline
&Classic Mistakes&Steve McConnell's list of classic mistakes \cite{McConnell1996}\\ \hline
2&Core Concepts&The project triangle, process and productive work, the cone of uncertainty, etc.\\ \hline
&Teams&Concepts (trust, conflict, accountability, etc.), leadership types, Tuckman's \cite{Tuckman1965} model\\ \hline
3&Risk Management&What/when/why, assessment and control, quantification, risk tables\\ \hline
&Lifecycle Planning&Explanations of 10 basic models: waterfall, spiral, evolutionary delivery, etc.  \\ \hline
4&Methodology Concepts&Cockburn's methodology structure \cite{cockburn1997}, plan-driven methodologies (PSP, TSP, RUP)\\ \hline
5&PM Anti-Patterns&What an anti-pattern is and some major examples \cite{antipatterns2014}\\ \hline
6&Agile Methodologies&Agile concepts, comparisons to plan-driven, specific methodologies (Scrum, etc)\\ \hline
7&Estimation&Basic process, challenges, methods, LOC vs. function points, risk reserves, expectations\\ \hline
&CoCoMo&Constructive Cost Model \cite{Boehm1981} use cases, calculation, benefit, limitations, w/ function points\\ \hline
8&Scheduling&Scheduling concepts, work breakdown structures, sequencing, scheduling tools, tracking\\ \hline
9&Quality&Definition, relationship with testing, verification and validation, quality assurance\\ \hline
10&Measurement/Metrics&Defined, project/product/process metrics, collection, analysis, examples of metrics\\ \hline
11&Testing&Testing concepts, sweet spot, pooling/seeding/etc., the V model \cite{Feiler2012}\\ \hline
12&Change Control&Types of change, maintenance (types, concepts), metrics, software distribution\\ \hline
13&Communications&Channels, planning, stakeholders and team communication, professional responsibility\\ \hline
14&Process Quality&Maturity models (CMMI \cite{cmmi2014}), process frameworks, application of changes to processes\\ \hline\end{tabular}
\end{table*}

\section{Introduction}
Process is a major focus of software engineering and its curriculum \cite{Tadayon:2004:SEB:1050231.1050248,107024}; because of this, project management has
been included as a required topic of study in many of these programs \cite{Walker:2002:IRT:775742.775765}. While
project management principles and practices are frequently a part of these classes \cite{mit2012,Texas2012,wvu2014,rit2014}, many do not include the
opportunity to participate as a project manager (PM) or as a member of a PM-led team \cite{Kessler:2007:ITA:1227504.1227420}. It is important to include the hands-on leadership and planning
elements that make project management a discipline rather than simply conveying a collection of related
methodologies \cite{1612134}. In many cases, the disciplines involved in
project management itself has fallen to the instructors; this is often carried out either through frequent
direct intervention with student groups (i.e.\ leadership) or through heavily structured assignment descriptions (i.e.\ project plans).
Unfortunately, this may serve to negate the need or desire of individual students to venture into realistic
project management within group work scenarios. As a result, these classes may be neglecting the lessons and skills 
that all computing students need in a realistic team environment.

At \ifisnopii the Rochester Institute of Technology (RIT)\else XXXX\fi, we have offered an upper division Process and
Project Management class within the Software Engineering major since \ifisnopii 1998\else XXXX\fi. The focus of this course
has included process methodologies, team development, and project management fundamentals. A
project component has always been a significant part of this course, but until this point its primary focus
has been delivery of project artifacts. In this paper, we describe an innovative approach for including a
hands-on project management experience within the project component of the course. Under the
supervision of the instructor, who serves as an advisor, students are given the opportunity to volunteer as
PMs for the main group project. These PMs are given traditional expectations in
managing their group's deliverables and dynamics, but are also expected to participate in a separate PM-only
group that enhances their learning experience as well as that of their team members.

This updated project format has been included in at least five class offerings and has experienced
substantial success. Students have stated that it not only increased their knowledge and application of
project management as a discipline, but that it has given them an opportunity to interact with project managers
as a group member or vice versa. Results, in many cases, have far exceeded expectations, and student
feedback has shown praise for both the interactive nature of the project and the resulting final
presentation.

The remainder of this paper is organized as follows. Section~\ref{sec:course} includes a description of the Process
and Project Management course, including its purpose, structure, and components. Section~\ref{sec:project} describes
the project, including its former state and changes that have been made to meet specific learning
objectives. Section~\ref{sec:results} shares the results, including examples from recent implementations of the project.
 Section~\ref{sec:feedback} includes both quantitative and qualitative student feedback. Section~\ref{sec:related} references
related work. Section~\ref{sec:future} discusses planned or proposed future work related to the project, and section~\ref{sec:summary}
provides a summary.

\section{About the Course}
\label{sec:course}
Although students are primarily Software Engineering majors, Process and Project Management is
also offered to other majors, including Computer Science, Computer Engineering, and Game Design. The
only prerequisite is the Introduction to Software Engineering course, a survey course which includes basic
concepts core to the major, such as requirements gathering, design, patterns, the concept of quality, and 
the engineer's focus on identifying and solving the problem. In this prerequisite, students have also been introduced to some of the themes of
Process and Project Management as well: teamwork and roles, an introduction to software development
process methodologies, and basic scheduling and task management.

Three of the primary goals of this course are to introduce students to the core concepts and artifacts of
project management, to continue to reinforce the software engineering process including process models,
and to demonstrate the importance of process and project management in the students' chosen discipline.
Lectures and texts enhance the concepts with case studies and real-world examples, striving for both
present and future relevance. In addition to process, covered concepts include classic mistakes (and
anti-patterns), team development, specific software engineering models (waterfall, agile, etc.), risk
management, estimating and scheduling, quality and metrics, communication management, and process
maturity models. Table \ref{table:topics} includes a schedule of topics covered in this 15 week course.

The Software Engineering department considers this 3-credit course the core class in the process track
(one of two major tracks) taken by all students in the major. This class is a prerequisite for other classes,
such as Software Process and Product Quality, and Trends in Software Development Process.
Methodologies and processes taught in this class are also a required implementation in the Senior Project
class which immediately precedes graduation. The department understands that a strong foundation in
this area is a vital part of students' future success and the reputation of the college.

Software Engineering majors typically take this course in their third year, and it often directly
proceeds or follows students' required one-year cooperative internship (co-op). For many students, this
time period is a watershed moment, as upper level courses and co-ops often have the effect of
encouraging the student to realize their area of focus and concentration. Though not always an explicit
minor, students naturally begin to specialize in areas such as testing, design, enterprise or web systems,
process and project management, or other related disciplines.

While most students are not likely to become PMs directly upon graduation, we do expend effort to allow students to see the value of
the discipline and its individual practices, which will inevitably come into play in the modern team-based computing environment. Half of class time is devoted to lectures, and the
remainder is reserved for reinforcing activities, discussion, and group work time. Students are graded in
several criteria, including short quizzes, three exams, individual and group activities, and a large group
project. Class sizes have typically ranged from 20 to 35 students.

\section{About the Project}
\label{sec:project}
This course has always had a major project component, as exposure to both the expectations and the artifacts
within a typical project has been an objective since its inception.  This project has been in many ways similar to those in other classes: 
groups are assigned, each group is required to complete a paper, and all are required to present findings to the
class at the end of the term. The primary deliverable is a project plan based upon a problem
statement provided by the instructor early in the course.

\begin{table}
\label{table:projectsched}
\caption{Project Activity by Week}
\begin{tabular}{|p{0.3cm}|p{2.5cm}|p{4.4cm}|} \hline
\cellcolor[gray]{0.9}&\textbf{Activity}\cellcolor[gray]{0.9}&\textbf{Details}\cellcolor[gray]{0.9}\\ \hline
1-5&Pre-Project&Students are encouraged to review the project outline\\ \hline
6&Project Begins&Required deliverables and due dates set\\ \hline
9&Draft 1 Due&Outline, risks, scope, requirements\\ \hline
12&Peer Evaluation 1&\\ \hline
12&Draft 2 Due&Updates to draft 1, process methodology, estimating, and scheduling\\ \hline
13&Cross-Group Feedback&Feedback effort is graded\\ \hline
14&Final Version Due&Updates to draft 2, lessons learned\\ \hline
15&Group Presentations&10-15 minutes in length\\ \hline
15&Peer Evaluation 2&Completed after final presentation\\ 
\hline\end{tabular}
\end{table}

While the problem statement has varied, the deliverables have remained consistent: an overview and scope, list of
functional and nonfunctional requirements, methodologies overview, schedules and their justifications,
risks, metrics, and lessons learned. Deliverables are turned in three times, with each building on the
previous version. Groups participate in cross-team feedback with other groups, and a 10-15 minute final
presentation takes place during the last week of the semester. Opportunities for group members to
provide feedback on each other's performance are in week 12 and at the end of the semester. Table 2 
contains the main activities and their typical timetable.

Because of its similarity to other paper-based group projects, students have been familiar with and
competent at completing the assignment, but many have felt that it was merely an extension of individual
assignments and have treated it as such. It had become evident that student groups have been dividing work
ineffectively and inconsistencies in both the content and flow of their papers and the final presentation
have demonstrated this ineffectiveness. These symptoms and the desire to allow students to have a PM-led 
experience have prompted us to make some changes to both the project and its disposition.

The first significant change is the inclusion of a formal PM role within the group project. Students
are notified on several occasions prior to beginning the project that the final project teams are to be led by
a voluntary PM. At the same time, students are told that this PM will have the opportunity to earn a
higher grade; peer evaluations are a significant part of the grade, and positive leadership as a PM is a
good way to earn higher evaluations. Those who are considering volunteering are asked to review the PM
Activity Guide, a document that specifies their responsibilities as a PM. Finally, they are asked to note
preferred team members for an opportunity to be afforded to them in group assignment efforts later in the
semester.  Group assignments early in the semester, in-class activities, and previous interactions with other
students are useful in assisting with evaluation of potential team members.

Selection of the PMs takes place at the start of the project directly after the first midterm, roughly one
third of the way through the semester. The process is public by show of hands and is continued until the
appropriate number of PMs have volunteered. Students and instructors are rarely surprised at who has
chosen to volunteer, as many have worked together in previous classes or even in the early part of the
current class. In the past, there have always been an appropriate number of volunteers, and rarely have
any volunteered who did not receive the opportunity to participate as a PM.  In this situation, a random subset
is chosen by the instructor. Previous efforts have yielded
between 1/4 and 1/5 of the class --- an appropriate number, as 4 or 5 students per group is desirable.

The second change has been to treat the PMs as a separate group, requiring them to cooperate in
several separate activities. The first activity exclusive to this group is the formation of the teams that they
will each lead. This takes place immediately after selection of PMs and is a private negotiation
process between PMs, as not to embarrass team members who are chosen near the end. As the semester
progresses, PMs are called together weekly to check progress, answer questions about upcoming
deliverables, and to mutually benefit each other in these exchanges. Checking attendance is integrated as
well; PMs are asked if any of their group members are missing, and, if so, whether they had indicated to
the group their expected absence. At the end of the semester, PMs are required to evaluate each other in
the areas of teamwork, knowledge and skills, dependability, initiative and creativity, adaptability and
flexibility, and delivery of results. Table \ref{table:PMtasks} contains the main activities and their typical timetable.

\begin{table}
\label{table:PMtasks}
\caption{Project Manager Activity by Week}
\begin{tabular}{|p{0.3cm}|p{2.9cm}|p{4.0cm}|} \hline
\cellcolor[gray]{0.9}&\textbf{Activity}\cellcolor[gray]{0.9}&\textbf{PM Responsibility}\cellcolor[gray]{0.9}\\ \hline
1-5&Consideration&Potential PMs consider volunteering\\ \hline
6&Project Begins&Volunteer as PM, final roster selection\\ \hline
7-11&Weekly Check-Ins&Cross-team problem solving\\ \hline
9-14&Deliverables Due&Manage group schedule, division of work, and accountability\\ \hline
12-14&Presentation Differentiation&PMs meet at least twice, provide summary to the instructor\\ \hline
15&Group Presentations&Report Order of presentations to the instructor\\ \hline
15&PM Peer Evaluation&Completed after final presentation\\ 
\hline\end{tabular}
\end{table}

The final and possibly the most unique change to the project relates directly to the separate PM-only
group. As a group, the PMs are expected to initiate a way of differentiating the final presentation.
Because each group is completing a project with the same guidelines, case study, and deliverable, the
final presentations can be both repetitive and rather difficult to grade, with later-presenting groups
unfairly benefiting from the insights or mistakes of their predecessors. Relating to their task of
differentiation, some guidelines and previous examples are given, but the task is intentionally left up to
the PMs. They are required to meet twice near the end of the semester and to provide a meeting summary
to the instructor.

Benefits to this differentiation are seen in both the presentation itself and the engagement of the
students both before and after the presentation. Because of the requirement to differentiate, 
group members are forced to prepare something other than a rehash of their paper.  During the
presentation itself students are more likely to listen, participate, and learn because the other groups'
presentations are each significantly different. Although the project deliverables do not extend beyond
project documentation, the opportunity to create something unique in the final presentation can act as a de
facto product for the team, giving them the satisfaction of creating something besides an unimplemented
project plan.

\section{Project Results}
\label{sec:results}
Class dynamics have been positive since the implementation of the project changes. The grouping of
students as the application and combining of concepts becomes a more prominent part of the course has
allowed students to participate in class activities as larger units rather than individual students. The
instructor has been able to call on groups rather than individuals to answer a question, resulting in less
individual embarrassment or awkward class flow and in a more positive cooperative effort.

Because of the group selection technique, instances of a ``super group'' or a ``left-over'' group formed after others 
have banded together has become less common.  Although there are still instances of
groups that perform significantly better or worse than their peers, final grade distribution typically indicates
that groups have a good mixture of students. In many instances, the PMs apply the team building principles
learned in the first part of the course not only to group management, but also in consideration and
selection of the team members themselves. Overall, this has resulted in more diverse, and therefore more
consistently successful, groups.

The experience within the group project has also had positive effects on the students individually. In
many cases, students have discovered or cemented a desire to pursue project management as their chosen
field, and have attributed that choice at least in part to the class project experience. Additionally, many
students have reported that lessons learned within their group were immediately applicable in co-ops or
other classes, and viewed group work differently than they had previously. Both PMs and group members
have indicated that the experience also made them better team members, as they had a greater knowledge
of the responsibilities of a PM and were able to assist in ways they previously had not even considered.
These results have been in line with pedagogical goals, especially demonstrating the importance of process 
and project management in the academic and work environment.

Diversification of the final presentation has also had surprising effects. The PM groups, tasked with
working together to make the final presentation more interesting and less repetitive, have come up with
some very innovative ways of doing this. Some of the best results have come from simple ideas like
combining all groups slides into one deck for presentation --- eliminating much of the downtime between
presentations and some of the unfair advantage that later presenting groups hold over their predecessors. PMs have also
served as timekeepers for other groups, monitored their team members to ensure they are paying attention, and 
have reviewed each other's planned presentation against the published rubric beforehand.

The most typical method of final presentation diversification has been to either divide by subject area
(i.e.\ risks, methodology, etc.) or to focus more on what each group has done differently rather than
repeating similar parts of their project implementation. The most surprising and imaginative
result so far has been a project management play depicting the project's progress through its planning
stages --- including 5 minutes in Shakespearian English, video projector sets, and a ``process methodology
smack-down''. In all cases, the resulting presentations have been more interesting and have required
students to be more engaged in both the preparation and disposition of their contribution.

Student feedback has been generally positive, and is discussed in the next section.

\begin{table*}[t]
\caption{Survey questions and results (\% who agree/strongly agree) from PMs and group members}
\label{table:satisfaction}
\begin{tabular}{|p{16.3cm}|c|} \hline
\textbf{The Field of Project Management}\cellcolor[gray]{0.9}&\cellcolor[gray]{0.9}\\ \hline
The use of project managers in this course enhanced my understanding of project
management as a discipline&85\%\\ \hline
The use of project managers has increased my interest in the field of project
management&85\%\\ \hline
\textbf{Project Manager-Led Groups}\cellcolor[gray]{0.9}&\cellcolor[gray]{0.9}\\ \hline
The project manager group made time management and transitions between
presentations easier or less intrusive&91\%\\ \hline
The opportunity to participate as a project manager increased my overall satisfaction with the course (even if I did not choose to participate as a project manager)&73\%\\ \hline
Overall, the use of an assigned project manager improved group dynamics&84\%\\ \hline
Overall, the use of an assigned project manager made my group project more successful&91\%\\ \hline
\textbf{Diversification of the Final Presentation}\cellcolor[gray]{0.9}&\cellcolor[gray]{0.9}\\ \hline
I feel that I learned more from diversification of the groups' presentations than I would have if each group had presented similar material&87\%\\ \hline
My preparation and engagement for the presentation was more interesting because of diversification of the groups' presentations&87\%\\ \hline
Other groups' presentations were more engaging because of diversification of the groups' presentations&82\%\\ \hline\end{tabular}
\end{table*}

\section{Student Feedback}
\label{sec:feedback}
Students have expressed high satisfaction with various elements of the group project within the
course. In a voluntary survey given at the end of the semester, students were asked to compare previous
group work issues with those encountered during this class. Issues reported as previously common but
reduced for the duration of this project included poor time management and organization, lack of
leadership, complications with division of labor, communication breakdown, and failure of teammates to
show up to meetings.

The survey also asked for general feedback on the group project. Some of their responses were as
follows:

\begin{quotation}I really like how the project managers volunteered for the position, because it meant that they
were willing to put forth the effort to manage the group, and as a result I felt more motivated to
participate as a member.\end{quotation}

\begin{quotation}The use of project managers helped keep our group on track, moving forward and not waiting
until the last minute to start working on each section.\end{quotation}

\begin{quotation}The project managers were helpful because it gave our group a certain line of communication
with the professor, which was more helpful than individually having questions answered.\end{quotation}

\begin{quotation}I liked the idea of all of us presenting one big presentation with each group in charge of a
specific part.\end{quotation}

\begin{quotation}I think the use of project managers really helped highlight the things we were learning in this
class --- at least that was the experience I had in my group. When you have a proactive PM who is
good about getting people to show up to meetings and actually getting their work done, it becomes
much easier to complete a project, and do it well.\end{quotation}

Students were also asked questions related to learning, project success, and engagement with the field
of project management. Questions were answered using a standard Likert scale. Table \ref{table:satisfaction} lists
statements and the percentages that agreed or strongly agreed. Respondents comprised of 90\% or greater of classes
surveyed. 21\% of respondents participated as a PM.

In general, students who volunteered to lead a group as a PM were more engaged, stated that they
learned more, and expressed greater satisfaction with the project. Students who did not choose to
participate as a PM also seemed to have an improved experience, and in some cases have stated that they
would like to lead project teams in future classes.

\section{Related Work}
\label{sec:related}
There has been significant development in the areas of both process and project management in the
classroom. Previous works have stated the importance of such an educational focus and, although varied, 
they lend credibility to providing a more realistic, PM-led team experience in the classroom. Oudshoorn, Brown, 
and Maciunas \cite{533976} discussed implementation of a more
realistic problem solving situations for software engineering project teams. Similarly, Villarreal and
Butler \cite{Villarreal:1998:GCS:274790.273157} and Henry and LaFrance \cite{Henry:2006:IRS:1181901.1181908} 
emphasized the importance of realistic experience and pioneered methodologies in this
area, expressing the understanding that unrealistic classroom situations and projects do not provide as
much value as some may believe. Providing a more realistic teamwork experience in the software
engineering classroom has also been specifically focused upon by Walker and Slotterbeck \cite{Walker:2002:IRT:775742.775765},
showing the need to address the issue before students have reached their capstone class.

Tan and Phillips \cite{Tan:2005:RPM:1059888.1059947} outlined an example of bringing more realistic project management scenarios
into the computer information systems curriculum. A comparison of project management instruction
through heavy use of antipatterns verses patterns in instruction was the focus of research by Staemelos,
Settas, and Mallini \cite{6065055}. Goldin and Rudahl \cite{5341212}, Albernethy, Piegari, and Reichgelt \cite{Abernethy:2007:TPM:1181849.1181888}, and
Tan and Jones \cite{Tan:2008:CSC:1352627.1352651} have presented methodologies for presenting processes in such a way that they
become meaningful, such as an experience-based approach or having teams interact directly with clients
external to the classroom. Most of these authors have also included explanations of the additional
demands that are placed on the instructor, and have in many cases built upon each other's work. When considered 
as a group, they show a need for more direct engagement by students in the disposition of the project itself, 
rather than more passive preplanned instructor project management.

\section{Future Work}
\label{sec:future}
This updated project format has been successfully utilized in several sections of the Process and
Project Management course, but there are enhancements planned for future sections. Moving forward,
one of the main objectives is to provide a group project environment that more realistically simulates both
the actual and the ideal project in the real world soon to be encountered by the students. In relation to
this, the structure of the deliverables could be organized differently, with more guidance related to
individual parts, such as sample risks, less reliance upon the instructor to define what should be included
in functional and nonfunctional requirements, and the possible introduction of a mid-project requirements
change.

One risk that has so far not been encountered is a lack of or a surplus of volunteers for the role of PM.
This may require more explicit definitions of both the role and contingencies. The role the PM fulfills
within their group could also be more explicitly defined by requiring agendas, meeting minutes, and
lessons learned at regular intervals throughout the class.

Given that the PMs in the class are relatively inexperienced leaders, surprisingly few issues have been
encountered related to this. The negotiation process by the PMs to select team members is not well
documented and can vary with personalities and circumstances. The meetings between the PMs in
preparation for the final presentation have not encountered any issues, no group has expressed the wish to
expel their PM, and no PM has dropped the class or explicitly chosen to discontinue the role as of yet.
While these risks are minimal, mitigation and management strategies should be put in place should the
need arise.

Use of an explicit PM role and deliberate differentiation of the final presentation is something that
could be adapted for use in other courses, especially those that have similar projects conducted by
multiple groups.  As an example, in a class where multiple groups have solved the same problem, the
final presentation could, through interaction between groups, completely omit problem definition
and instead focus on the differences of the groups' results.


\section{Summary}
\label{sec:summary}
We feel that it is important for students, as part of a process-oriented study, to have the opportunity to
experience a PM-led team, either as a voluntary PM or as a team member. This experience could prove valuable 
to any computing student, because modern work environments frequently require team interaction, with or 
without a PM or team leader. In response to this, we have
developed an innovative project structure which not only fulfills this need but also serves to increase
variety and student attentiveness to the final group presentation.

We have witnessed an increase in student satisfaction, improved group dynamics, interest in the
field of project management, and a greater understanding of the modern team-driven computing environment. 
Instructors and surveyed students have noted that groups more thoroughly engage with the
project as well as the other students participating the final presentation. It is our sincere hope that others
will find the ideas and results outlined in this paper inspiring, possibly resulting in the choice to make
similar improvements to courses or academic programs in which they participate.

\balance
\bibliographystyle{abbrv}
\bibliography{PMs-in-Classroom}
\end{document}
