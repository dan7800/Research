Large software systems are typically created by numerous developers. Files may be created and maintained by a single developer, or through a collaborative effort by many. Code ownership may be defined as the individual(s) who are primarily responsible for the creation of an application component~\cite{Rahman:2011:OED:1985793.1985860}.

A single developer acting as the primary owner of a component has several potential benefits. The first is that code ownership leads to high quality results through pride of workmanship and is easier to hold individuals accountable for the creation and maintenance of the component. Research has also shown that more developers working on a single component typically leads to more defects~\cite{Raymond:2001:CBM:560911}. However, owned code may suffer due to lack of external review since there may be less people examining it and may therefore hinder the discovery of problems~\cite{Rahman:2011:OED:1985793.1985860}. 

In practice there are several different models of code ownership. As defined by fowler~\cite{fowler_codeownership} the three levels of code ownership range from strong include, strong weak, and collective. Archie extends this notion to Design Ownership and provides an infrastructure for the developers to sign a tactical spike and be accountable for developing, testing, extending and maintaining related code snippets.

All architecturally significant code which has been mapped to TTPs or automatically detected by Archie are monitored for change activity. \emph{Archie} integrates an event-based traceability engine where all the tactical classes are registered with the event server, and the owner of tactical spikes is registered as a subscriber. Whenever a user modifies architecturally significant code, a notification event is triggered and the owner is notified. The developer making changes is also sent a message in the warning list depicting a visual representation of the underlying architectural. Figure~\ref{fig:Monitoring} depicts what the developer sees on a class that implements heartbeat emitter functionality in Hadoop~\emph{dataNode}. 

 The monitoring system highlights all tactic-related code, using a different color for each tactic. In this example, the~\emph{heartbeat}-related code is highlighted. The heartbeat TTP is concurrently shown to the developer that code in~\emph{datanode.java} serves as the heartbeat emitter, and that it sends heartbeats to~\emph{HeartBeatManager}. Along with this refactoring a message is sent to the owner of Heartbeat tactic in the system. The owner is the developer who first implemented the tactic or identified themselves as the owner.
