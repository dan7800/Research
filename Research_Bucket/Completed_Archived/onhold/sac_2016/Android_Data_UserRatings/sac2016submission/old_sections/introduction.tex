\section{Introduction}

Over the last decade the use of smart mobile devices has exploded. Mobile apps are the software applications that run on mobile devices such as smart phones and tablets. These apps are often distributed to the end users through a centralized online store. There are currently millions of apps in these online stores (eg. Google Play app-store\footnote{Google Play app-store: \url{play.google.com/store/apps}} and Apple's App Store\footnote{Apple's App Store: \url{www.apple.com/iphone/apps-for-iphone}}) and Android users download more than 1.5 billion applications apps from GooglePlay every month\footnote{\url{http://developer.android.com/about/}}. Apps are a major part of mobile consumer technology and have changed the computing experience of our modern digital society, allowing users to perform a variety of tasks not previously possible in a portable environment.


% developed by hundreds of thousands of developers that are being used by millions of end users. Estimates suggest that end users download billions of apps each year and total value is placed at billions of dollars. \todo{add citations for these.} 





The success of a mobile app is largely determined by user ratings from a digital storefront (`app store'). Users expect apps to continually provide new features, otherwise poor app store reviews and low ratings can be expected~\cite{Khalid2014}. Developers must frequently update their app's dependencies to keep up with the rapid progress of mobile technology~\cite{Syer2013}. Most importantly, apps must be secure since they are a crucial entry point into our digital lives.

At a glance, one may assume that the challenge of security and customer satisfaction are trade-offs, since if developers focus on new features to keep the ratings up, security testing may slip. New security-inspired features may also be perceived by users as cumbersome, leading to lower ratings. Even a vulnerability in a dependency can be detrimental to users, yet developers may not have the resources to thoroughly inspect a third-party framework for security concerns. Experts have even warned~\cite{McGrawBSS} that security trade-offs with other properties such as usability and performance are considered universal.

But is this trade-off historically true in the case of mobile apps? Empirically, do mobile apps with higher ratings have more potential security risks? Or, do app store ratings represent a more all-encompassing measure of customer experience that indicates a maturity in all of the properties of an app, with security being just one aspect? These questions motivated us to empirically examine the relationship between user ratings and security. To measure potential security risks, we use automated static analysis tools specifically tailored to the Android platform. While far from a comprehensive security audit, the static analysis tools provide a broad and consistent measure of basic security flaws that might plague Android apps. To measure user rating, we extracted the user ratings of more than 1600\todo{update} Android apps from the Google Play app store. 

%%% Should probably briefly mention here how we did an apples to apples comparison of apps of the same genre, size etc.... - We tried to get things as similar as possible.



%Automated static analysis tools can help developers alleviate some basic security challenges by providing a simple method of evaluating potential security risks. Recently, tools such as AndroGuard and\todo{(OTHERS)} have been developed to evaluate Android-specific concerns, such as over-privileged code or misuse of intents. These tools can also provide researchers with a way of evaluating potential security risks of thousands of Android apps to discover trends. 

\emph{The objective of this study is to investigate the the relationship between potential security risks and customer satisfaction by empirically evaluating Android apps with static analysis tools}. Specifically, our research question is: \textit{Are user ratings correlated with low potential security risks and security permissions, or do apps with higher ratings have more security risks?}

We found that while lower rated applications requested more permissions, and certain security risk metrics were higher in them, most of the security risks were greater in higher rated applications. Based on our empirical evidence, we conclude that user ratings (which captures the user's perception of an app) is an all-encompassing metric that is not yet affected by higher security risks. \todo{fix this}

The rest of the paper is organized as follows: In Section~\ref{sec:studydesign} we present the design of our case study, where we explain what tools we use, what data we collect and how we collect it. In Section~\ref{sec:results}, we present the results of our case study. In Section~\ref{sec:relatedwork}, we discuss the related work. In Section~\ref{sec:threats}, we present the threats to validity for our study. And finally in Section~\ref{sec:conclusion}, we conclude the paper based on our findings. \todo{make sure this section is up to date}
%\begin{itemize}
  %\item RQ\#1: Are user ratings correlated with low potential security risks and security permissions?
  %\item RQ\#2: Are user ratings correlated security permissions?
%\end{itemize}
%\dan{bold these?}

%We reverse-engineered XXXX Android apps from the GooglePlay store, ran XXX static analsysis tools on each app. We aggregating the warnings and security risk measurements, and conducted statistical association tests between static analysis warnings and user ratings. We (DID OTHER ANALYSIS??)



% Rename this section?
%\subsection{Research Questions}

%By analyzing a large and diverse set of Android applications with user ratings above and below average~\emph{we want to empirically examine the relationship between user ratings and the security vulnerability, over privileges and requested permissions of the app.} Specifically, we seek to answer the following research questions: \\

%\textbf{}


%We found that apps with a rating of less than 3 had a higher rate of over privileges and security vulnerability score as compared to apps with a user rating of above 3. This indicates that security does impact the user's experience and ratings of apps.


%In this case study, we examine the effects that security metrics such as the vulnerability score and over and under privileges have on the user ratings of the application. We examined 1,898 applications from the GooglePlay store; with 1,000 having a rating of over 3, and 898 with a rating of under 3. \\

%\textbf{RQ\#2: Does the amount of security permissions given to the app affect user ratings?}

%When installing an application, users are asked to grant the application to access various permissions such as sending SMS messages, reading contacts or access the internet. Users are typically wary of applications which ask for a large number of permissions, especially ones which the application does not need~\cite{Lin:2012:EPU:2370216.2370290}.
% Might be able to find a better citation than this



%We found that lower rated applications requested more permissions than their higher rated counterparts.
% Mention how many permissions

%\todo{explain this more?}