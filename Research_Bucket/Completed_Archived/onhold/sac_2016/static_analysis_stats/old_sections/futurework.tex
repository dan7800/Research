
While we have demonstrated some interesting results through the collection of over \hl{68,000} Android apps, future work may be conducted in several key areas. We only analyzed free apps, and an interesting study would be to compare the free and paid apps using similar studies as the ones we conducted. Future studies could also analyze how apps evolve over time through the examination of numerous released versions of the same app. Google's new OS, ``M'', received a massive permissions overhaul and work may be done to see how this new release affects how developers user permissions. Naturally, more apps can always be examined and with new apps being released on a daily basis, the process is never ending.
The website and dataset could also be improved by examining more Android apps

While the static analysis tools used in our work have been demonstrated to be highly effective in previous research, there is always room for improving the static analysis tools which used in analysis such as these. No matter how effective they are, static analysis tools are imperfect and can always be improved upon.

\dan{further development of static analysis tools. While what we had }


%   ? Adoption rates of new Android APis    - Might not be a great idea to include this
%   Further undestanding of what these values mean
%   Website enhancements


%%  What was said in evaluations that we can address here
